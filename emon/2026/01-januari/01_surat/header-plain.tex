%%%%%%%%%%%%%%%%%%%%%%%%%%%%%%%%%%%%%%%%%
% Long Lined Cover Letter
% LaTeX Template
% Version 2.0 (September 17, 2021)
%
% This template originates from:
% https://www.LaTeXTemplates.com
%
% Authors: Fanchao Chen
% (chenfc@fudan.edu.cn)
%
% Modified by: Rex Evan
% (rex.evan96@gmail.com)
%
% License:
% CC BY-NC-SA 4.0 (https://creativecommons.org/licenses/by-nc-sa/4.0/)
%
%%%%%%%%%%%%%%%%%%%%%%%%%%%%%%%%%%%%%%%%%

%----------------------------------------------------------------------------------------
%	PACKAGES AND OTHER DOCUMENT CONFIGURATIONS
%----------------------------------------------------------------------------------------

\documentclass{article}

%\usepackage{charter} % Use the Charter font


\usepackage[
	a4paper,    % Paper size
	top=2.5cm,    % Top margin
	bottom=2.5cm, % Bottom margin
	left=2.5cm,   % Left margin
	right=2.5cm,  % Right margin
	%showframe  % Uncomment to show frames around the margins for debugging purposes
]{geometry}

\setlength{\parindent}{0pt}     % Paragraph indentation
\setlength{\parskip}{1em}       % Vertical space between paragraphs

\usepackage{graphicx}       % Required for including images
\usepackage{multirow}		% butuh untuk merge cell di tabel
\usepackage{fancyhdr}       % Required for customizing headers and footers

\fancypagestyle{firstpage}{%
	\fancyhf{} % Clear default headers/footers
	\renewcommand{\headrulewidth}{0pt} % No header rule
%	\renewcommand{\footrulewidth}{1pt} % Footer rule thickness
}

%	\fancyhf{} % Clear default headers/footers
%	\renewcommand{\headrulewidth}{1pt} % Header rule thickness
%	\renewcommand{\footrulewidth}{1pt} % Footer rule thickness
%}

\AtBeginDocument{\thispagestyle{firstpage}} % Use the first page headers/footers style on the first page
%\pagestyle{subsequentpages} % Use the subsequent pages headers/footers style on subsequent pages

%----------------------------------------------------------------------------------------

\begin{document}

%----------------------------------------------------------------------------------------
%	FIRST PAGE HEADER
%----------------------------------------------------------------------------------------

%\font\nullfont=cmr10

\begin{tabular}{ll}
   \multirow{4}{*}{\includegraphics[width=0.13\textwidth]{logo/bps-logo.png}} % Logo
   &
   \large{\textbf{\textit{BADAN PUSAT STATISTIK}}} \\
   & \large{\textbf{\textit{KABUPATEN SIKKA}}} \\
   & Jl. Wairklau Nomor 29 Maumere 86112, Telp (0382) 21371 \\
   & \textit{Homepage: sikkakab.bps.go.id, e-mail: bps5310@bps.go.id} \\
\end{tabular}

\hfill

\vspace{-1em} % Pull the rule closer to the logo

\rule{\linewidth}{1pt} % Horizontal rule

%----------------------------------------------------------------------------------------
%	PERIKOP SURAT
%----------------------------------------------------------------------------------------

\hfill
Maumere, 25 Maret 2025

\begin{tabular}{@{} lcl}
	Nomor&:&B-xxx/53106/KP.390/2025 \\
	Sifat&:&biasa \\
	Lampiran&:&1 (satu) lembar\\
    Perihal&:& Pemilihan \textit{Employee of the Month} BPS Kabupaten Sikka\\
\end{tabular}

\bigskip % Vertical whitespace

%----------------------------------------------------------------------------------------
%	ISI SURAT
%----------------------------------------------------------------------------------------

Ini adalah isi surat.
Silahkan ubah ini sesuai kebutuhan.
Walaupun di dalam text edit, tulisan ini dibuah per baris,
namun di dalam dokumen pdfnya, tulisan ini akan tetap dalam 1 paragraf.

Jika ingin membuat paragraf baru, maka harus ditambah 1 line kosong, a.k.a 1 kali tombol enter.
Mungkin beberapa orang akan bertanya, mengapa harus membuat template latex?
bukankah sudah ada MS Office, LibreOffice, atau bahkan Google Docs untuk membuat dokumen sederhana.
Apalagi dokumennya sesederhana surat seperti ini.


Jawabannya, karena saya tertarik mempelajarinya. :)

Demikian surat ini dibuat untuk digunakan sebagaimana mestinya.

\bigskip % Vertical whitespace

%----------------------------------------------------------------------------------------
%	TTD
%----------------------------------------------------------------------------------------
\bigskip
\hfill
\begin{tabular}{@{}c}
	Kepala Badan Pusat Statistik\\
    Kabupaten Sikka \bigskip\\
    \\
    \\
    \\
    \\
	Kristanto Setyo Utomo, SST, M.Si
\end{tabular}


\end{document}
